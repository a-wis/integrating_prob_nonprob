% Options for packages loaded elsewhere
\PassOptionsToPackage{unicode}{hyperref}
\PassOptionsToPackage{hyphens}{url}
%
\documentclass[
]{article}
\usepackage{lmodern}
\usepackage{amsmath}
\usepackage{ifxetex,ifluatex}
\ifnum 0\ifxetex 1\fi\ifluatex 1\fi=0 % if pdftex
  \usepackage[T1]{fontenc}
  \usepackage[utf8]{inputenc}
  \usepackage{textcomp} % provide euro and other symbols
  \usepackage{amssymb}
\else % if luatex or xetex
  \usepackage{unicode-math}
  \defaultfontfeatures{Scale=MatchLowercase}
  \defaultfontfeatures[\rmfamily]{Ligatures=TeX,Scale=1}
\fi
% Use upquote if available, for straight quotes in verbatim environments
\IfFileExists{upquote.sty}{\usepackage{upquote}}{}
\IfFileExists{microtype.sty}{% use microtype if available
  \usepackage[]{microtype}
  \UseMicrotypeSet[protrusion]{basicmath} % disable protrusion for tt fonts
}{}
\makeatletter
\@ifundefined{KOMAClassName}{% if non-KOMA class
  \IfFileExists{parskip.sty}{%
    \usepackage{parskip}
  }{% else
    \setlength{\parindent}{0pt}
    \setlength{\parskip}{6pt plus 2pt minus 1pt}}
}{% if KOMA class
  \KOMAoptions{parskip=half}}
\makeatother
\usepackage{xcolor}
\IfFileExists{xurl.sty}{\usepackage{xurl}}{} % add URL line breaks if available
\IfFileExists{bookmark.sty}{\usepackage{bookmark}}{\usepackage{hyperref}}
\hypersetup{
  pdftitle={Integrating Probability and Nonprobability Samples for Survey Inference},
  pdfauthor={Arkadiusz Wiśniowski, Joseph W. Sakshaug, Diego A. Perez Ruiz, Annelies G. Blom},
  hidelinks,
  pdfcreator={LaTeX via pandoc}}
\urlstyle{same} % disable monospaced font for URLs
\usepackage[margin=1in]{geometry}
\usepackage{color}
\usepackage{fancyvrb}
\newcommand{\VerbBar}{|}
\newcommand{\VERB}{\Verb[commandchars=\\\{\}]}
\DefineVerbatimEnvironment{Highlighting}{Verbatim}{commandchars=\\\{\}}
% Add ',fontsize=\small' for more characters per line
\usepackage{framed}
\definecolor{shadecolor}{RGB}{248,248,248}
\newenvironment{Shaded}{\begin{snugshade}}{\end{snugshade}}
\newcommand{\AlertTok}[1]{\textcolor[rgb]{0.94,0.16,0.16}{#1}}
\newcommand{\AnnotationTok}[1]{\textcolor[rgb]{0.56,0.35,0.01}{\textbf{\textit{#1}}}}
\newcommand{\AttributeTok}[1]{\textcolor[rgb]{0.77,0.63,0.00}{#1}}
\newcommand{\BaseNTok}[1]{\textcolor[rgb]{0.00,0.00,0.81}{#1}}
\newcommand{\BuiltInTok}[1]{#1}
\newcommand{\CharTok}[1]{\textcolor[rgb]{0.31,0.60,0.02}{#1}}
\newcommand{\CommentTok}[1]{\textcolor[rgb]{0.56,0.35,0.01}{\textit{#1}}}
\newcommand{\CommentVarTok}[1]{\textcolor[rgb]{0.56,0.35,0.01}{\textbf{\textit{#1}}}}
\newcommand{\ConstantTok}[1]{\textcolor[rgb]{0.00,0.00,0.00}{#1}}
\newcommand{\ControlFlowTok}[1]{\textcolor[rgb]{0.13,0.29,0.53}{\textbf{#1}}}
\newcommand{\DataTypeTok}[1]{\textcolor[rgb]{0.13,0.29,0.53}{#1}}
\newcommand{\DecValTok}[1]{\textcolor[rgb]{0.00,0.00,0.81}{#1}}
\newcommand{\DocumentationTok}[1]{\textcolor[rgb]{0.56,0.35,0.01}{\textbf{\textit{#1}}}}
\newcommand{\ErrorTok}[1]{\textcolor[rgb]{0.64,0.00,0.00}{\textbf{#1}}}
\newcommand{\ExtensionTok}[1]{#1}
\newcommand{\FloatTok}[1]{\textcolor[rgb]{0.00,0.00,0.81}{#1}}
\newcommand{\FunctionTok}[1]{\textcolor[rgb]{0.00,0.00,0.00}{#1}}
\newcommand{\ImportTok}[1]{#1}
\newcommand{\InformationTok}[1]{\textcolor[rgb]{0.56,0.35,0.01}{\textbf{\textit{#1}}}}
\newcommand{\KeywordTok}[1]{\textcolor[rgb]{0.13,0.29,0.53}{\textbf{#1}}}
\newcommand{\NormalTok}[1]{#1}
\newcommand{\OperatorTok}[1]{\textcolor[rgb]{0.81,0.36,0.00}{\textbf{#1}}}
\newcommand{\OtherTok}[1]{\textcolor[rgb]{0.56,0.35,0.01}{#1}}
\newcommand{\PreprocessorTok}[1]{\textcolor[rgb]{0.56,0.35,0.01}{\textit{#1}}}
\newcommand{\RegionMarkerTok}[1]{#1}
\newcommand{\SpecialCharTok}[1]{\textcolor[rgb]{0.00,0.00,0.00}{#1}}
\newcommand{\SpecialStringTok}[1]{\textcolor[rgb]{0.31,0.60,0.02}{#1}}
\newcommand{\StringTok}[1]{\textcolor[rgb]{0.31,0.60,0.02}{#1}}
\newcommand{\VariableTok}[1]{\textcolor[rgb]{0.00,0.00,0.00}{#1}}
\newcommand{\VerbatimStringTok}[1]{\textcolor[rgb]{0.31,0.60,0.02}{#1}}
\newcommand{\WarningTok}[1]{\textcolor[rgb]{0.56,0.35,0.01}{\textbf{\textit{#1}}}}
\usepackage{graphicx}
\makeatletter
\def\maxwidth{\ifdim\Gin@nat@width>\linewidth\linewidth\else\Gin@nat@width\fi}
\def\maxheight{\ifdim\Gin@nat@height>\textheight\textheight\else\Gin@nat@height\fi}
\makeatother
% Scale images if necessary, so that they will not overflow the page
% margins by default, and it is still possible to overwrite the defaults
% using explicit options in \includegraphics[width, height, ...]{}
\setkeys{Gin}{width=\maxwidth,height=\maxheight,keepaspectratio}
% Set default figure placement to htbp
\makeatletter
\def\fps@figure{htbp}
\makeatother
\setlength{\emergencystretch}{3em} % prevent overfull lines
\providecommand{\tightlist}{%
  \setlength{\itemsep}{0pt}\setlength{\parskip}{0pt}}
\setcounter{secnumdepth}{-\maxdimen} % remove section numbering
\ifluatex
  \usepackage{selnolig}  % disable illegal ligatures
\fi

\title{Integrating Probability and Nonprobability Samples for Survey
Inference}
\author{Arkadiusz Wiśniowski\footnote{Social Statistics Department,
  University of Manchester, United Kindgom; Contact:
  \href{mailto:a.wisniowski@manchester.ac.uk}{\nolinkurl{a.wisniowski@manchester.ac.uk}}},
Joseph W. Sakshaug\footnote{Department of Statistical Methods Research,
  Institute for Employment Research; Department of Statistics, Ludwig
  Maximilian University of Munich; and the School of Social Sciences,
  University of Mannheim, Germany}, Diego A. Perez Ruiz\footnote{Department
  of Mathematics, University of Manchester, United Kindgom}, Annelies G.
Blom\footnote{Collaborative Research Center 884 ``Political Economy of
  Reforms'' and the School of Social Sciences, University of Mannheim,
  Germany}}
\date{02 December 2020}

\begin{document}
\maketitle

\hypertarget{introduction}{%
\section{Introduction}\label{introduction}}

This document introduces computer code that reproduces the results of
estimating model parameters in a linear regression using Bayesian
inference. The estimation uses simulated probability data with prior
distributions constructed using simulated nonprobability data. It is
based on the method as proposed in:

Integrating Probability and Nonprobability Samples for Survey Inference,
\emph{Journal of Survey Statistics and Methodology}, Volume 8, Issue 1,
February 2020, Pages 120--147,
\url{https://doi.org/10.1093/jssam/smz051}.

The repository with the code is available at
\url{https://github.com/a-wis/integrating_prob_nonprob}.

\hypertarget{code}{%
\section{Code}\label{code}}

\hypertarget{required-packages}{%
\subsection{Required packages}\label{required-packages}}

\begin{Shaded}
\begin{Highlighting}[]
\FunctionTok{library}\NormalTok{(readr)}
\FunctionTok{library}\NormalTok{(tidyverse)}
\end{Highlighting}
\end{Shaded}

\hypertarget{generating-toy-data}{%
\subsection{Generating toy data}\label{generating-toy-data}}

The function \texttt{gen.sample()} generates a sample of size \texttt{n}
of a continuous response variable with the standard deviation
\texttt{sd\_response}. The mean is a linear equation given by parameters
\texttt{beta} (a vector including intercept) and two continuous
predictors with means provided in the vector \texttt{m\_x} and standard
deviations \texttt{sd\_cov1} and \texttt{sd\_cov2}, respectively. The
two predictors can also be correlated with correlation coefficient
\texttt{corr}. The response can be generated assuming a certain level of
\texttt{bias} in the second predictor (for generating a nonprobability
sample). Argument \texttt{bias} can also take a vector as an input.

\begin{Shaded}
\begin{Highlighting}[]
\NormalTok{gen.sample }\OtherTok{=} \ControlFlowTok{function}\NormalTok{(}\AttributeTok{beta =} \FunctionTok{c}\NormalTok{(}\DecValTok{1}\NormalTok{, }\FloatTok{0.5}\NormalTok{, }\FloatTok{0.1}\NormalTok{), }\AttributeTok{n =} \DecValTok{50}\NormalTok{, }\AttributeTok{m\_x =} \FunctionTok{c}\NormalTok{(}\DecValTok{0}\NormalTok{, }\DecValTok{5}\NormalTok{), }\AttributeTok{sd\_response =} \DecValTok{1}\NormalTok{, }
    \AttributeTok{sd\_cov1 =} \DecValTok{1}\NormalTok{, }\AttributeTok{sd\_cov2 =} \DecValTok{1}\NormalTok{, }\AttributeTok{corr =} \FloatTok{0.1}\NormalTok{, }\AttributeTok{bias =} \DecValTok{1}\NormalTok{) \{}
\NormalTok{    x1 }\OtherTok{=} \FunctionTok{rnorm}\NormalTok{(n, m\_x[}\DecValTok{1}\NormalTok{], sd\_cov1)  }\CommentTok{\#covariate 1}
\NormalTok{    x2\_m }\OtherTok{=}\NormalTok{ m\_x[}\DecValTok{2}\NormalTok{]  }\CommentTok{\# covariate2 mean}
\NormalTok{    x2 }\OtherTok{=}\NormalTok{ x2\_m }\SpecialCharTok{+}\NormalTok{ sd\_cov2}\SpecialCharTok{/}\NormalTok{sd\_cov1 }\SpecialCharTok{*}\NormalTok{ corr }\SpecialCharTok{*}\NormalTok{ (x1 }\SpecialCharTok{{-}}\NormalTok{ m\_x[}\DecValTok{1}\NormalTok{]) }\SpecialCharTok{+} \FunctionTok{rnorm}\NormalTok{(n, }\DecValTok{0}\NormalTok{, }\FunctionTok{sqrt}\NormalTok{(}\DecValTok{1} \SpecialCharTok{{-}}\NormalTok{ corr}\SpecialCharTok{\^{}}\DecValTok{2}\NormalTok{) }\SpecialCharTok{*} 
\NormalTok{        sd\_cov2)}
\NormalTok{    X }\OtherTok{=} \FunctionTok{matrix}\NormalTok{(}\FunctionTok{c}\NormalTok{(}\FunctionTok{rep}\NormalTok{(}\DecValTok{1}\NormalTok{, n), x1, x2), n, }\FunctionTok{length}\NormalTok{(beta))}
    \CommentTok{\# covariate No. 2 with correlation with cov1 if bias is specified as a vector,}
    \CommentTok{\# response is calculated for a given set of sampled X}
    \ControlFlowTok{if}\NormalTok{ (}\FunctionTok{length}\NormalTok{(bias) }\SpecialCharTok{==} \DecValTok{1}\NormalTok{) \{}
\NormalTok{        Y }\OtherTok{=}\NormalTok{ X }\SpecialCharTok{\%*\%}\NormalTok{ (beta }\SpecialCharTok{*}\NormalTok{ bias) }\SpecialCharTok{+} \FunctionTok{rnorm}\NormalTok{(n, }\DecValTok{0}\NormalTok{, sd\_response)}
\NormalTok{    \} }\ControlFlowTok{else}\NormalTok{ \{}
\NormalTok{        Y }\OtherTok{=}\NormalTok{ X }\SpecialCharTok{\%*\%} \FunctionTok{c}\NormalTok{(beta[}\SpecialCharTok{{-}}\DecValTok{3}\NormalTok{], beta[}\DecValTok{3}\NormalTok{] }\SpecialCharTok{*}\NormalTok{ bias[}\DecValTok{1}\NormalTok{]) }\SpecialCharTok{+} \FunctionTok{rnorm}\NormalTok{(n, }\DecValTok{0}\NormalTok{, sd\_response)}
        \ControlFlowTok{for}\NormalTok{ (i }\ControlFlowTok{in} \DecValTok{2}\SpecialCharTok{:}\FunctionTok{length}\NormalTok{(bias)) \{}
\NormalTok{            Y }\OtherTok{=} \FunctionTok{cbind}\NormalTok{(Y, X }\SpecialCharTok{\%*\%} \FunctionTok{c}\NormalTok{(beta[}\SpecialCharTok{{-}}\DecValTok{3}\NormalTok{], beta[}\DecValTok{3}\NormalTok{] }\SpecialCharTok{*}\NormalTok{ bias[i]) }\SpecialCharTok{+} \FunctionTok{rnorm}\NormalTok{(n, }\DecValTok{0}\NormalTok{, sd\_response))}
\NormalTok{        \}}
\NormalTok{    \}}
    \CommentTok{\# returns a list with response Y and predictors matrix X as an output}
    \FunctionTok{return}\NormalTok{(}\FunctionTok{list}\NormalTok{(}\AttributeTok{Y =}\NormalTok{ Y, }\AttributeTok{X =}\NormalTok{ X))}
\NormalTok{\}}
\end{Highlighting}
\end{Shaded}

\hypertarget{calculating-posterior-distribution}{%
\subsection{Calculating posterior
distribution}\label{calculating-posterior-distribution}}

This function calculates posterior mean and variance of the vector of
linear model coefficients and mean and variance of the precision
(inverse variance) of the linear regression model.

\begin{Shaded}
\begin{Highlighting}[]
\NormalTok{calc.posterior }\OtherTok{=} \ControlFlowTok{function}\NormalTok{(mu\_0, k\_0, }\AttributeTok{Vin =} \FunctionTok{diag}\NormalTok{(}\FunctionTok{length}\NormalTok{(mu\_0)), }\AttributeTok{a\_0 =} \DecValTok{0}\NormalTok{, }\AttributeTok{b\_0 =} \DecValTok{0}\NormalTok{, }
    \AttributeTok{y =}\NormalTok{ Y, }\AttributeTok{x =}\NormalTok{ X) \{}
    \ControlFlowTok{if}\NormalTok{ (}\FunctionTok{dim}\NormalTok{(}\FunctionTok{as.matrix}\NormalTok{(x))[}\DecValTok{2}\NormalTok{] }\SpecialCharTok{!=} \FunctionTok{length}\NormalTok{(mu\_0) }\SpecialCharTok{|} \FunctionTok{length}\NormalTok{(y) }\SpecialCharTok{!=} \FunctionTok{dim}\NormalTok{(}\FunctionTok{as.matrix}\NormalTok{(x))[}\DecValTok{1}\NormalTok{]) }
        \FunctionTok{print}\NormalTok{(}\StringTok{"Dimensions mismatch!"}\NormalTok{) }\ControlFlowTok{else}\NormalTok{ \{}
\NormalTok{        n\_mu\_0 }\OtherTok{=} \FunctionTok{length}\NormalTok{(mu\_0)}
\NormalTok{        p }\OtherTok{=} \FunctionTok{dim}\NormalTok{(}\FunctionTok{as.matrix}\NormalTok{(x))[}\DecValTok{2}\NormalTok{]}
\NormalTok{        n }\OtherTok{=} \FunctionTok{length}\NormalTok{(y)}
\NormalTok{        V }\OtherTok{=}\NormalTok{ Vin }\SpecialCharTok{*}\NormalTok{ k\_0}
        \CommentTok{\# some OLS values}
\NormalTok{        mu\_hat }\OtherTok{=} \FunctionTok{as.matrix}\NormalTok{(}\FunctionTok{lm}\NormalTok{(y }\SpecialCharTok{\textasciitilde{}}\NormalTok{ x }\SpecialCharTok{{-}} \DecValTok{1}\NormalTok{)}\SpecialCharTok{$}\NormalTok{coefficients)}
\NormalTok{        xtx }\OtherTok{=} \FunctionTok{t}\NormalTok{(x) }\SpecialCharTok{\%*\%}\NormalTok{ x}
\NormalTok{        RSS }\OtherTok{=} \FunctionTok{t}\NormalTok{(y }\SpecialCharTok{{-}}\NormalTok{ x }\SpecialCharTok{\%*\%}\NormalTok{ mu\_hat) }\SpecialCharTok{\%*\%}\NormalTok{ (y }\SpecialCharTok{{-}}\NormalTok{ x }\SpecialCharTok{\%*\%}\NormalTok{ mu\_hat)}
        \CommentTok{\# posterior mean of mu}
\NormalTok{        Sigma\_t }\OtherTok{=} \FunctionTok{solve}\NormalTok{(xtx }\SpecialCharTok{+} \FunctionTok{solve}\NormalTok{(V))}
\NormalTok{        W }\OtherTok{=}\NormalTok{ Sigma\_t }\SpecialCharTok{\%*\%}\NormalTok{ xtx}
\NormalTok{        mu\_mean }\OtherTok{=}\NormalTok{ W }\SpecialCharTok{\%*\%}\NormalTok{ mu\_hat }\SpecialCharTok{+}\NormalTok{ (}\FunctionTok{diag}\NormalTok{(p) }\SpecialCharTok{{-}}\NormalTok{ W) }\SpecialCharTok{\%*\%} \FunctionTok{as.matrix}\NormalTok{(mu\_0)}
        
\NormalTok{        SS }\OtherTok{=}\NormalTok{ RSS }\SpecialCharTok{+} \FunctionTok{t}\NormalTok{(mu\_hat }\SpecialCharTok{{-}} \FunctionTok{as.matrix}\NormalTok{(mu\_0)) }\SpecialCharTok{\%*\%} \FunctionTok{solve}\NormalTok{(}\FunctionTok{solve}\NormalTok{(xtx) }\SpecialCharTok{+}\NormalTok{ V) }\SpecialCharTok{\%*\%}\NormalTok{ (mu\_hat }\SpecialCharTok{{-}} 
            \FunctionTok{as.matrix}\NormalTok{(mu\_0))}
\NormalTok{        v }\OtherTok{=}\NormalTok{ n }\SpecialCharTok{+} \DecValTok{2} \SpecialCharTok{*}\NormalTok{ a\_0}
\NormalTok{        mu\_var }\OtherTok{=}\NormalTok{ Sigma\_t }\SpecialCharTok{*}\NormalTok{ (}\FunctionTok{as.numeric}\NormalTok{(SS) }\SpecialCharTok{+} \DecValTok{2} \SpecialCharTok{*}\NormalTok{ b\_0)}\SpecialCharTok{/}\NormalTok{(n }\SpecialCharTok{+} \DecValTok{2} \SpecialCharTok{*}\NormalTok{ a\_0 }\SpecialCharTok{{-}} \DecValTok{2}\NormalTok{)}
        \CommentTok{\# posterior of tau}
\NormalTok{        tau\_mean }\OtherTok{=}\NormalTok{ (n}\SpecialCharTok{/}\DecValTok{2} \SpecialCharTok{+}\NormalTok{ a\_0)}\SpecialCharTok{/}\NormalTok{(SS}\SpecialCharTok{/}\DecValTok{2} \SpecialCharTok{+}\NormalTok{ b\_0)}
\NormalTok{        tau\_var }\OtherTok{=}\NormalTok{ (n}\SpecialCharTok{/}\DecValTok{2} \SpecialCharTok{+}\NormalTok{ a\_0)}\SpecialCharTok{/}\NormalTok{(SS}\SpecialCharTok{/}\DecValTok{2} \SpecialCharTok{+}\NormalTok{ b\_0)}\SpecialCharTok{\^{}}\DecValTok{2}
        \FunctionTok{return}\NormalTok{(}\FunctionTok{list}\NormalTok{(}\AttributeTok{mu\_mean =}\NormalTok{ mu\_mean, }\AttributeTok{mu\_cov =} \FunctionTok{sqrt}\NormalTok{(}\FunctionTok{diag}\NormalTok{(mu\_var)), }\AttributeTok{tau\_mean\_sd =} \FunctionTok{c}\NormalTok{(tau\_mean, }
            \FunctionTok{sqrt}\NormalTok{(tau\_var))))}
\NormalTok{    \}}
\NormalTok{\}}
\end{Highlighting}
\end{Shaded}

\hypertarget{priors-functions}{%
\subsection{Priors functions}\label{priors-functions}}

These functions represent the assumptions of the prior distributions
constructed using nonprobability data (Section 2.1 of the article).

\begin{Shaded}
\begin{Highlighting}[]
\CommentTok{\# Conjugate, Eq. 14}
\NormalTok{fun.hot.c}\OtherTok{\textless{}{-}}\ControlFlowTok{function}\NormalTok{(hot.n,n)\{}\ControlFlowTok{if}\NormalTok{ (hot.n }\SpecialCharTok{\textless{}} \FloatTok{0.05}\NormalTok{) }\DecValTok{1}\SpecialCharTok{/}\FunctionTok{log}\NormalTok{(n) }\ControlFlowTok{else} \DecValTok{1}\SpecialCharTok{/}\NormalTok{n\}   }
\CommentTok{\# conjugate{-}distance, Eq. 16}
\NormalTok{fun.diff.c1}\OtherTok{\textless{}{-}}\ControlFlowTok{function}\NormalTok{(bp,bnp,snp)\{}\FunctionTok{diag}\NormalTok{(}\FunctionTok{pmax}\NormalTok{((bp}\SpecialCharTok{{-}}\NormalTok{bnp)}\SpecialCharTok{\^{}}\DecValTok{2}\NormalTok{,snp}\SpecialCharTok{\^{}}\DecValTok{2}\NormalTok{))\}     }
\CommentTok{\# Zellner, Eq. 15}
\NormalTok{fun.hot.z2}\OtherTok{\textless{}{-}}\ControlFlowTok{function}\NormalTok{(hot.n,n)\{}\ControlFlowTok{if}\NormalTok{ (hot.n }\SpecialCharTok{\textless{}} \FloatTok{0.05}\NormalTok{) (n}\SpecialCharTok{\^{}}\DecValTok{2}\NormalTok{) }\ControlFlowTok{else} \DecValTok{1}\NormalTok{\} }
\CommentTok{\# Zellner{-}distance, Eq.17}
\NormalTok{fun.diff.z1}\OtherTok{\textless{}{-}}\ControlFlowTok{function}\NormalTok{(bp,bnp,snp)\{}\FunctionTok{sqrt}\NormalTok{(}\FunctionTok{diag}\NormalTok{(}\FunctionTok{pmax}\NormalTok{((bp}\SpecialCharTok{{-}}\NormalTok{bnp)}\SpecialCharTok{\^{}}\DecValTok{2}\NormalTok{,snp}\SpecialCharTok{\^{}}\DecValTok{2}\NormalTok{)))\}     }
\end{Highlighting}
\end{Shaded}

\hypertarget{hotellings-test}{%
\subsection{Hotelling's test}\label{hotellings-test}}

Hotelling's test is used in calculating the posterior.

\begin{Shaded}
\begin{Highlighting}[]
\NormalTok{hotelling.test }\OtherTok{=}\ControlFlowTok{function}\NormalTok{(lm\_prob, lm\_np)\{}
\NormalTok{  xbar }\OtherTok{=}\NormalTok{ lm\_prob}\SpecialCharTok{$}\NormalTok{coefficients}
\NormalTok{  mu\_0 }\OtherTok{=}\NormalTok{ lm\_np}\SpecialCharTok{$}\NormalTok{coefficients}
\NormalTok{  vcovm }\OtherTok{=} \FunctionTok{vcov}\NormalTok{(lm\_prob)}
  
\NormalTok{  p }\OtherTok{=} \FunctionTok{length}\NormalTok{(lm\_prob}\SpecialCharTok{$}\NormalTok{coefficients)}
\NormalTok{  n }\OtherTok{=} \FunctionTok{length}\NormalTok{(lm\_prob}\SpecialCharTok{$}\NormalTok{residuals)}
  
\NormalTok{  t2 }\OtherTok{=} \FunctionTok{t}\NormalTok{(xbar }\SpecialCharTok{{-}}\NormalTok{ mu\_0)}\SpecialCharTok{\%*\%}\FunctionTok{solve}\NormalTok{(vcovm)}\SpecialCharTok{\%*\%}\NormalTok{(xbar }\SpecialCharTok{{-}}\NormalTok{ mu\_0)}
\NormalTok{  f }\OtherTok{=}\NormalTok{ p}\SpecialCharTok{*}\NormalTok{(n}\DecValTok{{-}1}\NormalTok{)}\SpecialCharTok{/}\NormalTok{(n}\SpecialCharTok{{-}}\NormalTok{p)}
\NormalTok{  Fstat }\OtherTok{=}\NormalTok{ t2}\SpecialCharTok{/}\NormalTok{f}
\NormalTok{  p\_val }\OtherTok{=} \FunctionTok{pf}\NormalTok{(Fstat,p,n}\SpecialCharTok{{-}}\NormalTok{p,}\AttributeTok{lower.tail =}\NormalTok{ F)}
  \FunctionTok{return}\NormalTok{(p\_val)}
\NormalTok{\}}
\end{Highlighting}
\end{Shaded}

\hypertarget{applying-the-method}{%
\section{Applying the method}\label{applying-the-method}}

This function takes an input of two lists (such as those resulting from
the \texttt{gen.sample()} function). The lists represent probability
(\texttt{sample\_prob}) and nonprobability (\texttt{sample\_np}) data.
Each list has two elements: \texttt{Y} being the response variable, and
\texttt{X} - the predictors matrix with a column of ones for intercept.

\begin{Shaded}
\begin{Highlighting}[]
\NormalTok{gen.res }\OtherTok{=} \ControlFlowTok{function}\NormalTok{(sample\_prob,sample\_np)\{}
  \CommentTok{\#sample\_prob {-} sample of prob data}
  \CommentTok{\#sample of nprob data}
  \CommentTok{\# conjugate non{-}inf result}
\NormalTok{  yp\_std}\OtherTok{=}\NormalTok{sample\_prob}\SpecialCharTok{$}\NormalTok{Y}
\NormalTok{  xp\_std}\OtherTok{=}\NormalTok{sample\_prob}\SpecialCharTok{$}\NormalTok{X}
\NormalTok{  coefsize}\OtherTok{=}\FunctionTok{dim}\NormalTok{(xp\_std)[}\DecValTok{2}\NormalTok{] }\CommentTok{\#number of coefficients}
\NormalTok{  k}\OtherTok{=}\FunctionTok{length}\NormalTok{(yp\_std) }\CommentTok{\# length of Prob sample}
\NormalTok{  result\_ni}\OtherTok{=}\FunctionTok{calc.posterior}\NormalTok{(}\AttributeTok{mu\_0=}\FunctionTok{rep}\NormalTok{(}\DecValTok{0}\NormalTok{,}\FunctionTok{dim}\NormalTok{(xp\_std)[}\DecValTok{2}\NormalTok{]), }\AttributeTok{k\_0=}\NormalTok{k, }\AttributeTok{y=}\NormalTok{yp\_std, }\AttributeTok{x=}\NormalTok{xp\_std)}
  \CommentTok{\#ML prob data}
\NormalTok{  lm\_pobj }\OtherTok{=} \FunctionTok{lm}\NormalTok{(yp\_std }\SpecialCharTok{\textasciitilde{}}\NormalTok{ xp\_std}\DecValTok{{-}1}\NormalTok{)}
  
  \CommentTok{\# nonprob data}
\NormalTok{  ynp\_std}\OtherTok{=}\NormalTok{sample\_np}\SpecialCharTok{$}\NormalTok{Y}
\NormalTok{  xnp\_std}\OtherTok{=}\NormalTok{sample\_np}\SpecialCharTok{$}\NormalTok{X}
\NormalTok{  nnp}\OtherTok{=}\FunctionTok{length}\NormalTok{(ynp\_std) }\CommentTok{\#np sample length}
  
  \CommentTok{\#ML for nonprob }
\NormalTok{  lm\_npobj }\OtherTok{=} \FunctionTok{lm}\NormalTok{(ynp\_std }\SpecialCharTok{\textasciitilde{}}\NormalTok{ xnp\_std}\DecValTok{{-}1}\NormalTok{)}
  \CommentTok{\#hotelling test}
\NormalTok{  hot.n }\OtherTok{=} \FunctionTok{hotelling.test}\NormalTok{(lm\_pobj,lm\_npobj)}
  
  \CommentTok{\#conjugate posterior \#k\_0fun.hot.c(hot.n,nnp)}
\NormalTok{  result\_c }\OtherTok{=} \FunctionTok{calc.posterior}\NormalTok{(}\AttributeTok{mu\_0 =}\NormalTok{ lm\_npobj}\SpecialCharTok{$}\NormalTok{coefficients, }
                            \AttributeTok{k\_0 =} \FunctionTok{fun.hot.c}\NormalTok{(hot.n,nnp), }\AttributeTok{y=}\NormalTok{yp\_std,}\AttributeTok{x=}\NormalTok{xp\_std)}
  \CommentTok{\#conjugate difference posterior}
  \CommentTok{\# fun.diff.c1(lm\_pobj$coefficients, }
  \CommentTok{\#             lm\_npobj$coefficients, }
  \CommentTok{\#             summary(lm\_npobj)$coefficients[,2])}
\NormalTok{  result\_cd }\OtherTok{=} \FunctionTok{calc.posterior}\NormalTok{(}\AttributeTok{mu\_0 =}\NormalTok{ lm\_npobj}\SpecialCharTok{$}\NormalTok{coefficients, }
                             \AttributeTok{k\_0 =} \DecValTok{1}\SpecialCharTok{/}\FunctionTok{log}\NormalTok{(nnp), }
                             \AttributeTok{Vin=}\FunctionTok{fun.diff.c1}\NormalTok{(lm\_pobj}\SpecialCharTok{$}\NormalTok{coefficients, }
\NormalTok{                                             lm\_npobj}\SpecialCharTok{$}\NormalTok{coefficients, }
                                             \FunctionTok{summary}\NormalTok{(lm\_npobj)}\SpecialCharTok{$}\NormalTok{coefficients[,}\DecValTok{2}\NormalTok{]), }
                             \AttributeTok{y=}\NormalTok{yp\_std,}\AttributeTok{x=}\NormalTok{xp\_std) }\CommentTok{\#}
  \CommentTok{\#conjugate Zellner fun.hot.z2(hot.n,nnp)}
\NormalTok{  result\_z }\OtherTok{=} \FunctionTok{calc.posterior}\NormalTok{(}
  \AttributeTok{mu\_0 =}\NormalTok{ lm\_npobj}\SpecialCharTok{$}\NormalTok{coefficients, }
  \AttributeTok{k\_0 =} \FunctionTok{fun.hot.z2}\NormalTok{(hot.n,nnp), }\CommentTok{\#changed for the name}
  \AttributeTok{y=}\NormalTok{yp\_std,}\AttributeTok{x=}\NormalTok{xp\_std,}
  \AttributeTok{Vin =} \FunctionTok{solve}\NormalTok{(}\FunctionTok{t}\NormalTok{(xnp\_std)}\SpecialCharTok{\%*\%}\NormalTok{xnp\_std))}
  
  \CommentTok{\#conjugate Zellner difference fun.diff.z1}
\NormalTok{  result\_zd }\OtherTok{=} \FunctionTok{calc.posterior}\NormalTok{(}
  \AttributeTok{mu\_0 =}\NormalTok{ lm\_npobj}\SpecialCharTok{$}\NormalTok{coefficients, }
  \AttributeTok{k\_0 =}\NormalTok{ nnp,}
  \AttributeTok{y=}\NormalTok{yp\_std,}\AttributeTok{x=}\NormalTok{xp\_std,}
  \AttributeTok{Vin =} \FunctionTok{fun.diff.z1}\NormalTok{(lm\_pobj}\SpecialCharTok{$}\NormalTok{coefficients, }
\NormalTok{                    lm\_npobj}\SpecialCharTok{$}\NormalTok{coefficients,}
                    \FunctionTok{summary}\NormalTok{(lm\_npobj)}\SpecialCharTok{$}\NormalTok{coefficients[,}\DecValTok{2}\NormalTok{]) }\SpecialCharTok{\%*\%}
    \FunctionTok{solve}\NormalTok{(}\FunctionTok{t}\NormalTok{(xnp\_std)}\SpecialCharTok{\%*\%}\NormalTok{xnp\_std) }\SpecialCharTok{\%*\%} 
    \FunctionTok{fun.diff.z1}\NormalTok{(lm\_pobj}\SpecialCharTok{$}\NormalTok{coefficients,}
\NormalTok{                    lm\_npobj}\SpecialCharTok{$}\NormalTok{coefficients, }
                    \FunctionTok{summary}\NormalTok{(lm\_npobj)}\SpecialCharTok{$}\NormalTok{coefficients[,}\DecValTok{2}\NormalTok{]))}
  
  \CommentTok{\# wrapping results}
\NormalTok{  result }\OtherTok{=} \FunctionTok{rbind}\NormalTok{(}\FunctionTok{c}\NormalTok{(k,nnp,}
\NormalTok{                   result\_ni}\SpecialCharTok{$}\NormalTok{mu\_mean,result\_ni}\SpecialCharTok{$}\NormalTok{mu\_cov,}
\NormalTok{                   result\_c}\SpecialCharTok{$}\NormalTok{mu\_mean,result\_c}\SpecialCharTok{$}\NormalTok{mu\_cov,}
\NormalTok{                   result\_z}\SpecialCharTok{$}\NormalTok{mu\_mean,result\_z}\SpecialCharTok{$}\NormalTok{mu\_cov,}
\NormalTok{                   result\_cd}\SpecialCharTok{$}\NormalTok{mu\_mean,result\_cd}\SpecialCharTok{$}\NormalTok{mu\_cov,}
\NormalTok{                   result\_zd}\SpecialCharTok{$}\NormalTok{mu\_mean,result\_zd}\SpecialCharTok{$}\NormalTok{mu\_cov,}
\NormalTok{                   lm\_pobj}\SpecialCharTok{$}\NormalTok{coefficients,}\FunctionTok{summary}\NormalTok{(lm\_pobj)}\SpecialCharTok{$}\NormalTok{coefficients[,}\DecValTok{2}\NormalTok{],}
\NormalTok{                   lm\_npobj}\SpecialCharTok{$}\NormalTok{coefficients,}\FunctionTok{summary}\NormalTok{(lm\_npobj)}\SpecialCharTok{$}\NormalTok{coefficients[,}\DecValTok{2}\NormalTok{],hot.n)}
\NormalTok{            )}
  
  \CommentTok{\#saving as dataframe}
\NormalTok{  par.names }\OtherTok{=} \FunctionTok{paste0}\NormalTok{(}\StringTok{"Beta"}\NormalTok{,}\FunctionTok{c}\NormalTok{(}\DecValTok{1}\SpecialCharTok{:}\NormalTok{coefsize)) }
\NormalTok{  columns.name}\OtherTok{=}\FunctionTok{c}\NormalTok{(}\StringTok{"P\_ss"}\NormalTok{,}\StringTok{"NP\_ss"}\NormalTok{,}
                 \CommentTok{\#non{-}inf posterior estimates (mu=mean and se=standard dev)}
                 \FunctionTok{paste0}\NormalTok{(}\StringTok{"NI.mu."}\NormalTok{,par.names),}\FunctionTok{paste0}\NormalTok{(}\StringTok{"NI.se."}\NormalTok{,par.names),}
                 \CommentTok{\#conjugate}
                 \FunctionTok{paste0}\NormalTok{(}\StringTok{"C.mu."}\NormalTok{,par.names),}\FunctionTok{paste0}\NormalTok{(}\StringTok{"C.se."}\NormalTok{,par.names),}
                 \CommentTok{\#Zellner}
                 \FunctionTok{paste0}\NormalTok{(}\StringTok{"Z.mu."}\NormalTok{,par.names),}\FunctionTok{paste0}\NormalTok{(}\StringTok{"Z.se."}\NormalTok{,par.names),}
                 \CommentTok{\#conjugate{-}difference}
                 \FunctionTok{paste0}\NormalTok{(}\StringTok{"CD.mu."}\NormalTok{,par.names),}\FunctionTok{paste0}\NormalTok{(}\StringTok{"CD.se."}\NormalTok{,par.names),}
                 \CommentTok{\#Zellner difference}
                 \FunctionTok{paste0}\NormalTok{(}\StringTok{"ZD.mu."}\NormalTok{,par.names),}\FunctionTok{paste0}\NormalTok{(}\StringTok{"ZD.se."}\NormalTok{,par.names),}
                 \CommentTok{\#ML for probability data with standard errors}
                 \FunctionTok{paste0}\NormalTok{(}\StringTok{"MLP.mu."}\NormalTok{,par.names),}\FunctionTok{paste0}\NormalTok{(}\StringTok{"MLP.se."}\NormalTok{,par.names),}
                 \CommentTok{\#ML for non{-}probability data with standard errors}
                 \FunctionTok{paste0}\NormalTok{(}\StringTok{"MLNP.mu."}\NormalTok{,par.names),}\FunctionTok{paste0}\NormalTok{(}\StringTok{"MLNP.se."}\NormalTok{,par.names),}\StringTok{"Hotelling.p"}\NormalTok{)}
\NormalTok{  result}\OtherTok{=}\FunctionTok{as.data.frame}\NormalTok{(result)}
  \FunctionTok{colnames}\NormalTok{(result) }\OtherTok{=}\NormalTok{ columns.name}
  
  \FunctionTok{return}\NormalTok{(result)}
\NormalTok{\}}
\end{Highlighting}
\end{Shaded}

\hypertarget{toy-example}{%
\section{Toy Example}\label{toy-example}}

\hypertarget{generating-data}{%
\subsection{Generating data}\label{generating-data}}

Probability sample with parameters c(1, 0.5, .1) and size \(n=50\)

\begin{Shaded}
\begin{Highlighting}[]
\NormalTok{sample\_prob}\OtherTok{=}\FunctionTok{gen.sample}\NormalTok{(}\AttributeTok{beta=}\FunctionTok{c}\NormalTok{(}\DecValTok{1}\NormalTok{,}\FloatTok{0.5}\NormalTok{,.}\DecValTok{1}\NormalTok{), }\CommentTok{\#true coefficients}
                       \AttributeTok{n=}\DecValTok{50}\NormalTok{, }\CommentTok{\# sample size}
                       \AttributeTok{m\_x=}\FunctionTok{c}\NormalTok{(}\DecValTok{0}\NormalTok{,}\DecValTok{5}\NormalTok{), }\CommentTok{\#covariate means}
                       \AttributeTok{sd\_response=}\DecValTok{1}\NormalTok{, }\CommentTok{\#sd response variable}
                       \AttributeTok{sd\_cov1=}\DecValTok{1}\NormalTok{, }\AttributeTok{sd\_cov2=}\DecValTok{1}\NormalTok{, }\CommentTok{\# sd covariates}
                       \AttributeTok{corr=}\FloatTok{0.1}\NormalTok{, }\CommentTok{\#correcation of covariates}
                       \AttributeTok{bias=}\FunctionTok{c}\NormalTok{(}\DecValTok{1}\NormalTok{)) }\CommentTok{\#bias= 1 = unbiased}
\end{Highlighting}
\end{Shaded}

Probability sample with size \(n=1000\) and bias in the third
coefficient of \(+25\%\), i.e.~the data are generated with parameters
c(1, 0.5, .125)

\begin{Shaded}
\begin{Highlighting}[]
\NormalTok{sample\_np}\OtherTok{=}\FunctionTok{gen.sample}\NormalTok{(}\AttributeTok{beta=}\FunctionTok{c}\NormalTok{(}\DecValTok{1}\NormalTok{,}\FloatTok{0.5}\NormalTok{,.}\DecValTok{1}\NormalTok{), }
                     \AttributeTok{n=}\DecValTok{1000}\NormalTok{, }\CommentTok{\#sample size}
                     \AttributeTok{m\_x=}\FunctionTok{c}\NormalTok{(}\DecValTok{0}\NormalTok{,}\DecValTok{5}\NormalTok{), }\CommentTok{\#covariate means}
                     \AttributeTok{sd\_response=}\DecValTok{1}\NormalTok{, }\DocumentationTok{\#\#sd response variable}
                     \AttributeTok{sd\_cov1=}\DecValTok{1}\NormalTok{, }\AttributeTok{sd\_cov2=}\DecValTok{1}\NormalTok{,}\CommentTok{\# sd covariates}
                     \AttributeTok{corr=}\FloatTok{0.1}\NormalTok{,}\CommentTok{\#correcation of covariates}
                     \AttributeTok{bias=}\FunctionTok{c}\NormalTok{(}\FloatTok{1.25}\NormalTok{)) }\CommentTok{\#bias= 1 = unbiased = here, 25\% bias iin third coefficient}
\end{Highlighting}
\end{Shaded}

\hypertarget{producing-coefficients}{%
\subsection{Producing coefficients}\label{producing-coefficients}}

\begin{Shaded}
\begin{Highlighting}[]
\NormalTok{simA}\OtherTok{=}\FunctionTok{gen.res}\NormalTok{(sample\_prob,sample\_np)}
\end{Highlighting}
\end{Shaded}

\hypertarget{printing-results}{%
\subsection{Printing results}\label{printing-results}}

\begin{Shaded}
\begin{Highlighting}[]
\NormalTok{simA }\SpecialCharTok{\%\textgreater{}\%} \FunctionTok{pivot\_longer}\NormalTok{(}\AttributeTok{cols =}\NormalTok{ NI.mu.Beta1}\SpecialCharTok{:}\NormalTok{MLNP.se.Beta3, }\AttributeTok{names\_to =} \FunctionTok{c}\NormalTok{(}\StringTok{"method"}\NormalTok{, }\StringTok{"moment"}\NormalTok{, }
    \StringTok{"parameter"}\NormalTok{), }\AttributeTok{names\_sep =} \StringTok{"}\SpecialCharTok{\textbackslash{}\textbackslash{}}\StringTok{."}\NormalTok{, }\AttributeTok{values\_to =} \StringTok{"value"}\NormalTok{) }\SpecialCharTok{\%\textgreater{}\%} \FunctionTok{pivot\_wider}\NormalTok{(}\AttributeTok{names\_from =}\NormalTok{ moment, }
    \AttributeTok{values\_from =}\NormalTok{ value) }\SpecialCharTok{\%\textgreater{}\%}\NormalTok{ View}
\end{Highlighting}
\end{Shaded}

\hypertarget{saving-results-in-a-pasteable-table}{%
\subsection{Saving results in a pasteable
table}\label{saving-results-in-a-pasteable-table}}

\begin{Shaded}
\begin{Highlighting}[]
\NormalTok{simA }\SpecialCharTok{\%\textgreater{}\%} \FunctionTok{pivot\_longer}\NormalTok{(}\AttributeTok{cols =}\NormalTok{ NI.mu.Beta1}\SpecialCharTok{:}\NormalTok{MLNP.se.Beta3, }\AttributeTok{names\_to =} \FunctionTok{c}\NormalTok{(}\StringTok{"method"}\NormalTok{, }\StringTok{"moment"}\NormalTok{, }
    \StringTok{"parameter"}\NormalTok{), }\AttributeTok{names\_sep =} \StringTok{"}\SpecialCharTok{\textbackslash{}\textbackslash{}}\StringTok{."}\NormalTok{, }\AttributeTok{values\_to =} \StringTok{"value"}\NormalTok{) }\SpecialCharTok{\%\textgreater{}\%} \FunctionTok{pivot\_wider}\NormalTok{(}\AttributeTok{names\_from =}\NormalTok{ moment, }
    \AttributeTok{values\_from =}\NormalTok{ value) }\SpecialCharTok{\%\textgreater{}\%} \FunctionTok{write.table}\NormalTok{(}\AttributeTok{file =} \StringTok{"clipboard"}\NormalTok{)}
\end{Highlighting}
\end{Shaded}


\end{document}
